%%%%%%%%%%%%%%%%%%%%%%%%%%%%%%%%%%%%%%%%%%%%%%%%%%%%%%%%%%%%%%%%%%%%%%%%%%%%%%%%
% Arsclassica Article
% LaTeX Template
% Version 1.1 (10/6/14)
%
% This template has been downloaded from:
% http://www.LaTeXTemplates.com
%
% Original author:
% Lorenzo Pantieri (http://www.lorenzopantieri.net) with extensive modifications by:
% Vel (vel@latextemplates.com)
%
% License:
% CC BY-NC-SA 3.0 (http://creativecommons.org/licenses/by-nc-sa/3.0/)
%
%%%%%%%%%%%%%%%%%%%%%%%%%%%%%%%%%%%%%%%%%%%%%%%%%%%%%%%%%%%%%%%%%%%%%%%%%%%%%%%%

%-------------------------------------------------------------------------------
%	PACKAGES AND OTHER DOCUMENT CONFIGURATIONS
%-------------------------------------------------------------------------------

\documentclass[
10pt, % Main document font size
a4paper, % Paper type, use 'letterpaper' for US Letter paper
oneside, % One page layout (no page indentation)
%twoside, % Two page layout (page indentation for binding and different headers)
headinclude,footinclude, % Extra spacing for the header and footer
BCOR5mm, % Binding correction
]{scrartcl}

\input{structure.tex} % Include the structure.tex file which specified the document structure and layout

\hyphenation{Fortran hy-phen-ation} % Specify custom hyphenation points in words with dashes where you would like hyphenation to occur, or alternatively, don't put any dashes in a word to stop hyphenation altogether

%-------------------------------------------------------------------------------
%	TITLE AND AUTHOR(S)
%-------------------------------------------------------------------------------

\project{Métricas Estimativas}

\title{\normalfont\spacedallcaps{SISTEMA PASSE-LIVRE}} % The article title
%\title{Sistema de Apoio às Olimpíadas e Paraolimpíadas (SAPOP)} % The article title

%\author{\spacedlowsmallcaps{John Smith* \& James Smith\textsuperscript{1}}} % The article author(s) - author affiliations need to be specified in the AUTHOR AFFILIATIONS block
\author{\spacedlowsmallcaps{Jean Amaro (8532401), Wesley Tiozzo (8077925) \string& Danilo Zecchin Nery (8602430)}}

\date{\today} % An optional date to appear under the author(s)

%-------------------------------------------------------------------------------

\begin{document}

%-------------------------------------------------------------------------------
%	HEADERS
%-------------------------------------------------------------------------------

\renewcommand{\sectionmark}[1]{\markright{\spacedlowsmallcaps{#1}}} % The header for all pages (oneside) or for even pages (twoside)
%\renewcommand{\subsectionmark}[1]{\markright{\thesubsection~#1}} % Uncomment when using the twoside option - this modifies the header on odd pages
\lehead{\mbox{\llap{\small\thepage\kern1em\color{halfgray} \vline}\color{halfgray}\hspace{0.5em}\rightmark\hfil}} % The header style

\pagestyle{scrheadings} % Enable the headers specified in this block

%-------------------------------------------------------------------------------
%	TABLE OF CONTENTS & LISTS OF FIGURES AND TABLES
%-------------------------------------------------------------------------------

\maketitle % Print the title/author/date block

\tableofcontents % Print the table of contents

%\setcounter{tocdepth}{2} % Set the depth of the table of contents to show sections and subsections only

%\listoffigures % Print the list of figures

\listoftables % Print the list of tables

%-------------------------------------------------------------------------------
%	ABSTRACT
%-------------------------------------------------------------------------------

%\section*{Abstract} % This section will not appear in the table of contents due to the star (\section*)

%-------------------------------------------------------------------------------
%	AUTHOR AFFILIATIONS
%-------------------------------------------------------------------------------

%{\let\thefootnote\relax\footnotetext{* \textit{Department of Biology, University of Examples, London, United Kingdom}}}

%{\let\thefootnote\relax\footnotetext{\textsuperscript{1} \textit{Department of Chemistry, University of Examples, London, United Kingdom}}}

%-------------------------------------------------------------------------------

\newpage % Start the article content on the second page, remove this if you have a longer abstract that goes onto the second page

%-------------------------------------------------------------------------------
%	INTRODUCTION
%-------------------------------------------------------------------------------

\section{Introdução}\label{intro}
Este documento apresenta um estudo quantitativo do \textit{Sistema Passe-Livre},
com base num documento de requisitos e num ERD (\textit{Entity-Relationship Diagram})
fornecidos em especificação da tarefa.

A seção \ref{pfloc} apresenta o processo e o resultado final dos cálculos de PF
(\textit{Pontos por Função}) e LoC (\textit{Lines of Code}) para o sistema em
questão, hipoteticamente implementado em linguagem Java. A seção \ref{cdq}
apresenta os cálculos de Custo, Documentação e Qualidade para o sistema, com
base em valores arbitrários fornecidos em especificação. A seção \ref{etd}
apresenta os cálculos de esforço e tempo necessário para o desenvolvimento do
sistema, com uso do modelo \textit{Cocomo Básico}.

\newpage %%

%-------------------------------------------------------------------------------
%	PF + LoC
%-------------------------------------------------------------------------------

\section{Análise estimativa por PF \& LoC}\label{pfloc}

Com base no MER e Documento de Requisitos disponibilizados para este
trabalho, o grupo refletiu sobre como o sistema reagiria a cada evento.
Assim, para que o planejamento estivesse próximo ao código final, foram
criados os seguintes métodos para cada classe:
\begin{itemize}[noitemsep]
	\item proprietario 
		\begin{itemize}[noitemsep]
			\item \textit{adicionarGizmo}
			\item \textit{removerGizmo}
			\item \textit{consultarExtrato}
		\end{itemize}
	\item veiculo
		\begin{itemize}[noitemsep]
			\item \textit{adicionarGizmo}
			\item \textit{removerGizmo}
		\end{itemize}
	\item tagRFID
		\begin{itemize}[noitemsep]
			\item \textit{ativar}
			\item \textit{desativar}
		\end{itemize}
	\item sistemaFinaceiroOperadora 
		\begin{itemize}[noitemsep]
			\item \textit{enviarCobranca}
			\item \textit{emitirRelatorio}
		\end{itemize}
	\item leitorRFID 
		\begin{itemize}[noitemsep]
			\item enviarCodigo
			\item verificacaoAtivo
		\end{itemize}
	\item sistemaCentral 
		\begin{itemize}[noitemsep]
			\item \textit{verificacaoAtivo}
			\item \textit{calculaPedagio}
			\item \textit{gerarCobranca}
			\item \textit{adimplir}
			\item \textit{desamplir}
			\item \textit{enviarAviso}
		\end{itemize}
	\item sensorEixos
		\begin{itemize}[noitemsep]
			\item \textit{carroDentro}
		\end{itemize}
	\item cancela 
		\begin{itemize}[noitemsep]
			\item \textit{abre}
			\item \textit{fecha}
		\end{itemize}
\end{itemize}

Para a eventual implementação de cada classe presente no diagrama,
foram estimados os seguintes valores de linhas de código:
\begin{itemize}[noitemsep]
	\item 5 linhas para definição de um campo de variável (atributo,
		vetores, objetos, etc)
	\item 5 linhas para criação de setter deste campo
	\item 5 linhas para criação de getter deste campo
	\item 10 linhas para o construtor
	\item 20 linhas para a declaração e definição de cada método
	\item 20 linhas para a inserção de instruções em SQL, seja ela
		\textit{hard-coded} no código-fonte, ou obtida em fontes
		exteriores, caso haja a necessidade do método acessar o banco
		de dados
\end{itemize}

Além disso, como fazia parte do documento de requisitos a existência de acesso
via plataforma Web, foram estimadas as contagens de linhas de código para cada
página (incluindo estrutura, código JavaScript, stylesheets e chamadas ao
sistema). Considerando as informações supracitadas, observe as tabelas:
\begin{table}[!h]
	\begin{adjustwidth}{-1.4cm}{}
	\begin{tabular}{p{4cm} p{2.8cm} p{2cm} p{2cm} p{1.5cm}}
		\hline\hline
		Classe & Métodos & Instanciação da classe & Definição/ Declaração dos métodos & Acesso ao Banco de Dados\\ 
		\hline\hline
		proprietario & adicionarGizmo*, removerGizmo*, consultarExtrato* & 40 & 60 & 60\\ 
		\hline
		atendentePosto &  & 15 &  & \\ 
		\hline
		postoAutorizado &  & 25 &  & \\ 
		\hline
		operadora &  & 25 &  & \\ 
		\hline
		veiculo & adicionarGizmo*, removerGizmo* & 30 & 40 & 40\\ 
		\hline
		tagRFID & ativar*, desativar* & 40 & 40 & 40\\ 
		\hline
		sistemaFinaceiroOperadora & enviarCobranca*, emitirRelatorio* & 30 & 40 & 40\\ 
		\hline
		leitorRFID & enviarCodigo*, verificacaoAtivo & 30 & 40 & 20\\ 
		\hline
		sistemaCentral & verificacaoAtivo, calculaPedagio*, gerarCobranca*, adimplir*, desamplir*, enviarAviso* & 30 & 120 & 100\\ 
		\hline
		pedagio &  & 50 &  & \\ 
		\hline
		sensorEixos & carroDentro & 30 & 20 & \\ 
		\hline
		rodovia &  & 20 &  & \\ 
		\hline
		concessionaria &  & 15 &  & \\ 
		\hline
		cancela & abre/fecha & 25 & 40 & \\ 
		\hline
		funcionario &  & 20 &  & \\ 
		\hline
	\hline\end{tabular}
	\end{adjustwidth}
	\caption{\small{Tabela de Sistemas}} 
	\label{table:t1}
\end{table}

{\let\thefootnote\relax\footnotetext{* \textit{Métodos que dependem de um banco de dados implementado e funcional}}}

\begin{table}[!h]
	\centering
	\begin{tabular}{c c}
		\hline\hline
		Página & Linhas para criação da página\\ 
		\hline\hline
		Funcionario & 300\\ 
		\hline
		Cliente & 200\\ 
		\hline
		Tag & 100\\ 
		\hline
		Veiculo & 100\\ 
		\hline
		Administrador & 500\\ 
		\hline
		Analista Financeiro & 300\\ 
		\hline
	\hline\end{tabular}
	\caption{\small{Tabela de Plataforma Web}} 
	\label{table:t2}
\end{table}

\newpage %%

Foi utilizada métrica orientada função para esse trabalo. Sendo assim,
foram seguidos os três passos estudados em aula, presentes nos slides:
\begin{table}[!h]
	\begin{adjustwidth}{-1.1cm}{}
	\begin{tabular}{l | c | c c c | c}
		\hline\hline
		\multicolumn{1}{r}{ } & \multicolumn{1}{c}{Contagem} & Tipo & Contagem & \multicolumn{1}{c}{Peso} & Resultado\\ 
		\hline\hline
		\multirow{3}{*}{Entradas Externas} & \multirow{3}{*}{10} & Simples & 10 & 3 & \multirow{3}{*}{30}\\ 
		\cline{3-5}
		&  & Média & 0 & 4 & \\ 
		\cline{3-5}
		&  & Complexa & 0 & 6 & \\ 
		\hline
		\multirow{3}{*}{Saídas Externas} & \multirow{3}{*}{6} & Simples & 4 & 4 & \multirow{3}{*}{26}\\ 
		\cline{3-5}
		&  & Média & 2 & 5 & \\ 
		\cline{3-5}
		&  & Complexa & 0 & 7 & \\ 
		\hline
		\multirow{3}{*}{Consultas Externas} & \multirow{3}{*}{2} & Simples & 0 & 3 & \multirow{3}{*}{8}\\ 
		\cline{3-5}
		&  & Média & 2 & 4 & \\ 
		\cline{3-5}
		&  & Complexa & 0 & 6 & \\ 
		\hline
		\multirow{3}{*}{Arquivos Lógicos Internos} & \multirow{3}{*}{4} & Simples & 4 & 7 & \multirow{3}{*}{28}\\ 
		\cline{3-5}
		&  & Média & 0 & 10 & \\ 
		\cline{3-5}
		&  & Complexa & 0 & 15 & \\ 
		\hline
		\multirow{3}{*}{Arquivos de Interface Externa} & \multirow{3}{*}{10} & Simples & 10 & 5 & \multirow{3}{*}{50}\\ 
		\cline{3-5}
		&  & Média & 0 & 7 & \\ 
		\cline{3-5}
		&  & Complexa & 0 & 10 & \\ 
		\hline
		\multicolumn{5}{l}{PF Bruto} & 142\\ 
		\hline
	\hline\end{tabular}
	\end{adjustwidth}
	\caption{\small{Tabela do Passo 1}} 
	\label{table:t3}
\end{table}

\paragraph{Observação} Para a criação e preenchimento da tabela acima, foi
usada a seguinte lógica:
\begin{itemize}[noitemsep]
	\item Entradas Externas: Todas as classes que possuem atributo
	\item Saídas Externas: 4 para a criação, remoção, atualização e remoção (CRUD); 2 para geração de relatório
	\item Consultas Externas: \textit{consultarExtrato}, \textit{gerarRelatorio}
	\item Arquivos Lógicos Internos: Classes \textit{proprietario}, \textit{veiculo}, \textit{tagRFID}, \textit{funcionario}
	\item Arquivos de Interface Externa: Todas as classes que possuem atributo
\end{itemize}

\newpage %%

Após a criação da tabela \ref{table:t3}, foi respondido o todo questionário
presente em especificação, e chegamos aos seguintes pesos:
\begin{table}[!h]
	\centering
	\begin{tabular}{c c}
		\hline\hline
		Pergunta & Influência\\ 
		\hline\hline 
		1 & 5\\ 
		\hline
		2 & 3\\ 
		\hline
		3 & 1\\ 
		\hline
		4 & 5\\ 
		\hline
		5 & 0\\ 
		\hline
		6 & 5\\ 
		\hline
		7 & 1\\ 
		\hline
		8 & 4\\ 
		\hline
		9 & 2\\ 
		\hline
		10 & 2\\ 
		\hline
		11 & 2\\ 
		\hline
		12 & 1\\ 
		\hline
		13 & 4\\ 
		\hline
		14 & 5\\ 
		\hline
	\hline\end{tabular}
	\caption{\small{Tabela do Passo 2}} 
	\label{table:t4}
\end{table}

É, então, encontrado o fator de reajuste \textit{F} por meio de sua equação:
\begin{equation}
	F = 0,65 + 0,01 x \sum[setInfluences] = 1.05
\end{equation}
Onde \textit{setInfluences} é, claramente, o conjunto das influências
apresentadas na tabela \ref{table:t4}. Obtemos então, o PF ajustado:
\begin{table}[!h]
	\centering
	\begin{tabular}{l | c c}
		\hline\hline
		\multicolumn{1}{c}{ } & Resultado & PF ajustado\\ 
		\hline\hline
		\multirow{3}{*}{Entradas Externas} & \multirow{3}{*}{30} & \multirow{3}{*}{31,5}\\ 
		&  & \\ 
		&  & \\ 
		\hline
		\multirow{3}{*}{Saídas Externas} & \multirow{3}{*}{26} & \multirow{3}{*}{27,3}\\ 
		&  & \\ 
		&  & \\ 
		\hline
		\multirow{3}{*}{Consultas Externas} & \multirow{3}{*}{8} & \multirow{3}{*}{8,4}\\ 
		&  & \\ 
		&  & \\ 
		\hline
		\multirow{3}{*}{Arquivos Lógicos Internos} & \multirow{3}{*}{28} & \multirow{3}{*}{29,4}\\ 
		&  & \\ 
		&  & \\ 
		\hline
		\multirow{3}{*}{Arquivos de Interface Externa} & \multirow{3}{*}{50} & \multirow{3}{*}{52,5}\\ 
		&  & \\ 
		&  & \\ 
		\hline
		\multicolumn{1}{l}{PF Bruto} & 142 & 149,1\\ 
		\hline
	\hline\end{tabular}
	\caption{\small{Tabela do Passo 3}} 
	\label{table:t5}
\end{table}

\newpage %%

%-------------------------------------------------------------------------------
%	COST, DOC & Quality
%-------------------------------------------------------------------------------

\section{Dedução de Custo, Documentação e Qualidade}\label{cdq}
O Custo \textit{C} (Dólares), a Documentação \textit{D} (Páginas) e a
Qualidade \textit{Q} (Unidades de erro) foram calculados tendo como base as
fórmulas e valores hipotéticos apresentados em especificação:
\begin{equation}
	C = 23.0 * 149.1 = 3,429.3
\end{equation}
\begin{equation}
	D = 6.0 * 149.1 = 894.6
\end{equation}
\begin{equation}
	Q = 0.25 * 149.1 = 37.275
\end{equation}

\newpage %%

%-------------------------------------------------------------------------------
%	EFFORT, DEV TIME
%-------------------------------------------------------------------------------

\section{Esforço e Tempo de Desenvolvimento}\label{etd}
Para os valores de esforço \textit{E} (Pessoas-Mês) e tempo de desenvolvimento
\textit{T} (Meses), foram utilizadas as equações disponibilizadas em sala de
aula (sabendo que será utilizado o modelo \textit{COCOMO Básico}, com a classe
de projeto orgânico):
\begin{equation}
	E = 2.4 * 2.625^{1.05} = 6.61
\end{equation}
\begin{equation}
	T = 2.5 * 6.61^{0.38} = 5.12
\end{equation}

\newpage %%

%-------------------------------------------------------------------------------
%	CONCLUSION
%-------------------------------------------------------------------------------

\section{Conclusão}
Esta seção apresenta uma pequena inferência relativa aos objetivos propostos e
declarações feitas na seção \ref{intro}.

Os cálculos permitem fazer algumas presunções relativas ao projeto
a ser encarado, mas a precisão é certamente questionável. O que se busca sempre
é alguma garantia ou previsibilidade, e estes sempre são incertos. Uma abordagem
quantitativa para análise de qualidade, em perspectiva, não faz muito sentido
nem é tão satisfatória; não corresponde à realidade na maioria esmagadora dos
casos, o que é razoável.

Conclusões à respeito da qualidade de uma implantação são resultados de um
processo árduo e trabalhoso. Métodos como ponto por função e
\textit{Lines of Code}, sozinhos, não são respeitosos à qualidade real de um
projeto.

\end{document}
